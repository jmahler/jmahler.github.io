\documentclass{article}

\usepackage{graphviz}
\usepackage{fullpage}
\usepackage{parskip}
\usepackage{url}
\usepackage{hyperref}
\usepackage{appendix}

\usepackage{listings}
\lstset{
	numbers=left,
	basicstyle=\ttfamily\footnotesize,
	captionpos=b,
	xleftmargin=0.3in
}

\usepackage[
	backend=biber,
	autocite=footnote,
	bibstyle=authortitle,
	citestyle=verbose-inote,
]{biblatex}
\addbibresource{main.bib}
\setlength\bibitemsep{1em}

\begin{document}

\title{GitHub Actions Cookbook}
\author{
	\Large{Jeremiah Mahler}\\
	\small{\href{mailto:jmmahler@gmail.com}{\textless jmmahler@gmail.com\textgreater}}
}
\date{\today}
\maketitle
%\clearpage

\thispagestyle{empty}
\tableofcontents

\section{Introduction}

This is a cookbook of GitHub Actions \autocite{githubactions} recipes.

\section{Run Golang Unit Tests}

To run Golang unit tests in a GitHub Action simply run \verb+go test+
like you normally would \autocite{gotest} (Figure \ref{fig:gotest}).

\begin{figure}[!ht]
\begin{lstlisting}
    - name: Test
      run: |
        go test -coverprofile=profile.cov github.com/jmahler/rgm
        # test the cli but leave out coverage since it isn't useful
        go test github.com/jmahler/rgm/rgm
    - name: Integration Tests
      run: |
        go test -tags=integration github.com/jmahler/rgm
        go test -tags=integration github.com/jmahler/rgm/rgm
\end{lstlisting}
\caption{Golang Unit Tests in a GitHub Action.}
\label{fig:gotest}
\end{figure}

\section{TODO: Recipes}

Future recipes to be added.
Contributions are welcome.

\begin{itemize}
\item{Save .zip Build Artifacts}
\item{Build a LaTeX Document}
\item{Build a Docker Image}
\item{Save Artifacts to an S3 Bucket}
\item{Save Artifacts to BitBucket}
\item{Submit a Pull Request to a Second Git Repo}
\item{Trigger Action When Other Git Repo is Updated}
\item{Run Tests When a Dependent Repo is Updated}
\item{Build an RPM Package}
\item{Update an RPM Repo With a Built RPM Package}
\item{Update a GitHub.io Page}
\item{Update a Wiki Page}
\end{itemize}

\clearpage
\printbibliography[heading=bibintoc]

\end{document}
