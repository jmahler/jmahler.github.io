\documentclass{article}

\usepackage{graphviz}
\usepackage{fullpage}
\usepackage{parskip}
\usepackage{url}
\usepackage{hyperref}

\usepackage{listings}
\lstset{
	numbers=left,
	basicstyle=\footnotesize,
	captionpos=b,
	xleftmargin=0.3in
}

\usepackage[
	backend=biber,
	autocite=footnote,
	bibstyle=authortitle,
	citestyle=verbose-inote,
]{biblatex}
\addbibresource{main.bib}
\setlength\bibitemsep{1em}

\begin{document}

\title{Bug Hunt \#2: CUPS Successfully Prints Nothing}
\author{
	\Large{Jeremiah Mahler}\\
	\small{\href{mailto:jmmahler@gmail.com}{\textless jmmahler@gmail.com\textgreater}}
}
\date{June 5, 2015}
\maketitle
%\clearpage

%\thispagestyle{empty}
%\tableofcontents
%\clearpage

The first laptop prints without any issue.
The second laptop appears to print successfully but the printer
doesn't respond and no printed pages are output.

For all practical purposes, both machines are the same.
They both run the same version of Debian Linux and CUPS\autocite{cups}.
And their configurations are identical.

To debug this issue the debug logs can be used.  Each machine can
perform the same print operation and then the logs can be compared to
see if there are any differences.  The simplest operation is to
print a test page from the CUPS web interface.

Examining the logs for each machine finds that when the first (working)
machine begins sending data the second machine gets errors about
executables not being found (Figure \ref{fig:log}).

\begin{figure}[h!]
\begin{lstlisting}[breaklines=true]
(/var/log/cups/error_log)
...
[Job 4] PID 6827 (pstops) exited with no errors.
cupsd is not idle any more, canceling shutdown.
[Job 4] PID 6808 (/usr/lib/cups/filter/pdftops) exited with no errors.
cupsd is not idle any more, canceling shutdown.
[Job 4] /usr/lib/cups/filter/brother_lpdwrapper_mfcj825dw: 130: /usr/lib/cups/filter/brother_lpdwrapper_mfcj825dw: /opt/brother/Printers/mfcj825dw/cupswrapper/brcupsconfpt1: not found
cupsd is not idle any more, canceling shutdown.
[Job 4] /opt/brother/Printers/mfcj825dw/lpd/filtermfcj825dw: 67: /opt/brother/Printers/mfcj825dw/lpd/filtermfcj825dw: /opt/brother/Printers/mfcj825dw/lpd/brmfcj825dwfilter: not found
cupsd is not idle any more, canceling shutdown.
[Client 16] Accepted from localhost (Domain)
[Client 16] Waiting for request.
cupsd is not idle any more, canceling shutdown.
[Client 16] POST / HTTP/1.1
...
\end{lstlisting}
\caption{Output of /var/log/cups/error\_log when print fails.}
\label{fig:log}
\end{figure}

Checking to see if these files are missing finds that they are
actually present.
So why is saying they are not found?
It turns out that it is a problem with executables.
The Brother driver packages provide binaries for i386 but this is
an amd64 system.  The \verb+lib32gcc1+ package needs to be installed
to run these (\verb+sudo apt-get install lib32gcc1+).
And once this is done the second machine will print correctly.

This problem could have been prevented in several different ways.
First, is the return error in CUPS.  A ``not found'' error was in the
logs but this did not propagate to any place where a user would
notice.  Making it noticeable would have at least given the user
some clue that something is wrong.

The second way is to add \verb+lib32gcc1+ as a
requirement in the \verb+.deb+ package.  Unfortunately, Brother
controls their proprietary binary packages and there is no easy
way to report bugs or send them patches.
And since their package isn't included in Debian,
no one else can fix these bugs either.

%\clearpage
%\printbibliography[heading=bibintoc]

\end{document}
