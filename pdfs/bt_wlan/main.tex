\documentclass{article}

\usepackage{graphviz}
\usepackage{fullpage}
\usepackage{parskip}
\usepackage{url}
\usepackage{hyperref}

\usepackage{listings}
\lstset{
	numbers=left,
	basicstyle=\footnotesize,
	captionpos=b,
	xleftmargin=0.3in
}

\usepackage[
	backend=biber,
	autocite=footnote,
	bibstyle=authortitle,
	citestyle=verbose-inote,
]{biblatex}
\addbibresource{main.bib}
\setlength\bibitemsep{1em}

\begin{document}

\title{Bug Hunt \#4: Bluetooth, WLAN, Coexistence Fix}
\author{
	\Large{Jeremiah Mahler}\\
	\small{\href{mailto:jmmahler@gmail.com}{\textless jmmahler@gmail.com\textgreater}}
}
\date{\today}
\maketitle
%\clearpage

%\thispagestyle{empty}
%\tableofcontents
%\clearpage

I noticed a post on LKML about a problem that I thought was interesting
\autocite{lkml201564703}.
A person named Jonas Thiem was having problems using both the wireless lan
and the Bluetooth on his computer.
When they are both active he gets lots of errors and neither of them
work.
If he loads the \verb+iwlwifi+ module with the \verb+lln_disabled=1+
option this gets around the problem.
I agree with Jonas that, from a users point of view, they should both
just work.
A user should not have to fiddle with special driver options.

A little research on Bluetooth and WLAN finds that they both share
the frequency bands near 2.4 GHz\autocite{wiki:bt, wiki:802.11n}.
So it is certainly possible that they could be interfering with each
other.

It would be interesting to test whether a computer running only
Bluetooth would cause interference with a separate computer running
only wlan.
This would tell as whether it is due to wireless interference or some
issue specific to the machine.

A reply from Arend van Spriel\autocite{lkml201565137} mentioned the
Linux kernel work on ``bluetooth coexistence'' which is aimed at
addressing exactly this sort of issue.
Specifically, when both Bluetooth and WLAN are used on the same machine,
it can look at the frequencies being used and preemptively prevent
interference by notifying the other to use a different frequency
\autocite{bluetooth-coexistence}.

More research is needed to address this problem.
First, the problem needs to be reproduced on a machine that is available
for testing.
Second, a fix needs to be devised that aligns with the Bluetooth
coexistence schem that is already present in the kernel.

%\clearpage
%\printbibliography[heading=bibintoc]

\end{document}
