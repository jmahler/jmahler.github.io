\documentclass{article}

\usepackage{graphviz}
\usepackage{fullpage}
\usepackage{parskip}
\usepackage{url}
\usepackage{hyperref}
\usepackage{appendix}

\usepackage{listings}
\lstset{
	numbers=left,
	basicstyle=\ttfamily\footnotesize,
	captionpos=b,
	xleftmargin=1.0in
}

\usepackage[
	backend=biber,
	autocite=footnote,
	bibstyle=authortitle,
	citestyle=verbose-inote,
]{biblatex}
\addbibresource{main.bib}
\setlength\bibitemsep{1em}

\begin{document}

\title{Github Actions vs Gitlab CI}
\author{
	\Large{Jeremiah Mahler}\\
	\small{\href{mailto:jmmahler@gmail.com}{\textless jmmahler@gmail.com\textgreater}}
}
\date{\today}
\maketitle
%\clearpage

\thispagestyle{empty}
\tableofcontents

\section{Introduction}

Continuous Integration systems are widely used but the features
the support are diverse.
This paper compares two continuous integration systems: Github
Actions\autocite{github-actions} and Gitlab CI\autocite{gitlab-ci}.
To achieve real world comparisions a single project is used and
the same solution is implemented using both systems.

\section{Building a LaTeX Document}

Each of the systems has a different syntax for configuring the
steps to perform.
Both use YAML but the structure is different.
In this example a LaTeX document will be built and the artifact
stored so it is available online.

A first (naive) attempt at building a LaTeX document with Github Actions
involved several steps
(Figure
\ref{fig:github-actions-build})\autocite{github-jmahler-resume-build.yml}\autocite{github-resume-build-91697435}.
Using the Ubuntu 18.04 image, first the Git repo had to be checked out.
Since this is a generic image, the necessary LaTeX packages had
to be installed.
Make is run which runs pdflatex to build the docs.
And finally, the upload-artifact action is run to save the doc.

\begin{figure}[!ht]
\begin{lstlisting}
# .github/workflows/build.yml

name: Build
on: [push, pull_request]

jobs:
  build:
    name: Build
    runs-on: ubuntu-18.04
    steps:
    - name: Checkout
      uses: actions/checkout@v2
    - name: Install LaTeX
      run: |
        sudo apt install texlive-latex-base
    - name: Make Doc
      run: |
        make
    - name: Upload PDFs
      uses: actions/upload-artifact@v2
      with:
        name: linux-packaging
        path: linux-packaging.pdf
\end{lstlisting}
\label{fig:github-actions-build}
\caption{Build of a LaTeX doc using Github Actions.}
\end{figure}

Building a LaTeX document with Gitlab CI is much simpler
(Figure
\ref{fig:gitlab-ci-build})\autocite{gitlab-jmmahler-resume-gitlab-ci.yml}\autocite{gitlab-resume-build-533777462}.
The main reason for this simplification was the use of a
Docker image which is setup for building LaTeX documents.
The image was found in Gitlab by simply looking through their CI
templates.
Gitlab provides a wide assortment of templates with Docker images
ready to build practically anything.

\begin{figure}[!ht]
\begin{lstlisting}
# gitlab-ci.yml

image: blang/latex

build:
  script:
    - make
  artifacts:
    paths:
      - "*.pdf"
\end{lstlisting}
\label{fig:gitlab-ci-build}
\caption{Build of a LaTeX doc using GitLab CI.}
\end{figure}

Could the Github Actions implementation be simplified by using the
Docker image that Gitlab CI is using?
Not really.
The design of GitHub Actions is such that the Docker image must be
defined inside it's own \verb+-action+ repo
\autocite{github-action-docker}.
Then this action can be used from the workflow file
(Figure \ref{fig:github-action-with-docker}).
Conceptually, the worflow file controls the virtual machine that
is being run, the action controls the Docker image.

\begin{figure}[!ht]
\begin{lstlisting}
diff --git a/.github/workflows/build.yml b/.github/workflows/build.yml
index ed84053..36f8f3c 100644
--- a/.github/workflows/build.yml
+++ b/.github/workflows/build.yml
@@ -8,12 +8,10 @@ jobs:
     steps:
     - name: Checkout
       uses: actions/checkout@v2
-    - name: Install LaTeX
-      run: |
-        sudo apt install texlive-latex-base
     - name: Make Doc
-      run: |
-        make
+      uses: xu-cheng/latex-action@master
+      with:
+        root_file: linux-packaging.tex
     - name: Upload PDFs
       uses: actions/upload-artifact@v2
       with:
\end{lstlisting}
\label{fig:github-action-with-docker}
\caption{Changes needed to use Docker with a GitHub action.}
\end{figure}

For building a LaTeX document, GitLab is the simpler solution.
GitHub requires more setup and more boiler plate code to get
an equivalent solution working.

\clearpage
\printbibliography[heading=bibintoc]

\end{document}
